\documentclass[12pt]{article}
\usepackage{fullpage,color,graphicx}


%\documentclass[11pt]{article}
\usepackage{graphicx}
%\usepackage[colorlinks=true,urlcolor=blue]{hyperref}
\usepackage{sans}
\usepackage{color}

\newcommand{\pjm}[1]{{\color{blue}#1}}
\definecolor{MyGreen}{cmyk}{1.0,0.0,1.0,0.2}
\newcommand{\sjc}[1]{{\color{MyGreen}#1}}
\definecolor{MyFuschia}{cmyk}{0.07,0.95,0,0}
\newcommand{\ejm}[1]{{\color{MyFuschia}#1}}
\definecolor{bloodred}{RGB}{102,0,0}
\newcommand{\blood}[1]{{\color{bloodred}#1}}
\definecolor{mygrey}{gray}{0.45}
\newcommand{\grey}[1]{{\color{mygrey}#1}}
\definecolor{orange}{RGB}{255,102,0}
\newcommand{\orange}[1]{{\color{orange}#1}}
\newcommand{\red}[1]{{\color{red}#1}}
\newcommand{\blue}[1]{{\color{blue}#1}}
% The official Dartmouth green: 
\definecolor{dartmouthgreen}{rgb}{0,.41,.24}


%\usepackage{scicite}

\newcommand{\compresslist}{%
\setlength{\itemsep}{1pt}%
\setlength{\parskip}{0pt}%
\setlength{\parsep}{0pt}%
}

\oddsidemargin=0.25in
\evensidemargin=\oddsidemargin
\textwidth=6in
\topmargin=-0.5in
\textheight=9in

\begin{document}

\thispagestyle{empty}


\noindent \textbf{Point-by-point discussion of Reviews and Response to Comments}\\

First, we thank the editor...

\bigskip
\bigskip


\noindent \underline{\textbf{Response to the Editor's Comments}}\\

\noindent {\bf E1} \grey{\emph{Overall, the referees are quite supportive of the manuscript, however, they raise several issues that require attention. As at least two referees note, the manuscript needs to be geared to an audience that may not be familiar with text based analysis, thus more hand holding is needed, such as tucking the highly technical aspects into an appendix and incorporating a discussion of how to apply this technique (i.e., workflow) }}\\

\noindent \textcolor{MyGreen}{\textbf{Addressed:}} The Editor.. \\

\noindent  {\bf E2} \grey{\emph{In addition, including an application that demonstrates its value is necessary. One of the reviewers suggests a full blown theoretical contribution, but that is not necessary.  }}\\

\noindent \textcolor{MyGreen}{\textbf{Addressed:}} The Editor... \\


\noindent \underline{\textbf{Response to Comments by Reviewer 1}}\\

\noindent {\bf R1.1} \grey{\emph{ The shortcoming in my view is that there is no innovative example application. Other scholars are already using similar methods and state bills to investigate, for example, the role of interest groups in promoting policy diffusion and the state level variables that impact diffusion. So suggesting that these are potential uses of the method seems well... Can the authors think of any other innovative applications? After all, they now have millions of bill comparisons at their disposal.}}\\

\noindent \textcolor{MyGreen}{\textbf{Addressed:}} The reviewer... \\

\noindent {\bf R1.2} \grey{\emph{  A final sugestions is that much of the discussion of the method is directed at scholars who are familiar with text reuse methods (rather than the target audience) and is probably best reserved for an appendix (freeing up space for an example application!)}}\\

\noindent \textcolor{MyGreen}{\textbf{Addressed:}} The reviewer... \\

\noindent {\bf R1.3} \grey{\emph{  One other thought. They validate with the limited similar bills listed by the NCSL. One possible application might be to explore how many other bills meet that similarity threshold and how they are similar and different in terms of substance, states, sponsors etc to those identified by NCSL}}\\

\noindent \textcolor{MyGreen}{\textbf{Addressed:}} The reviewer... \\

\noindent \underline{\textbf{Response to Comments by Reviewer 2}}\\

\noindent {\bf R2.1} \grey{\emph{   I was unsure to what extent the algorithm used in this paper overlaps with the Smith-Waterman algorithm. The authors make clear the use the same algorithm but with some adjustments. What's part of the original algorithm and what's new was not clear to me. The authors should explain the difference more clearly.}}\\

\noindent \textcolor{MyGreen}{\textbf{Addressed:}} The reviewer... \\

\noindent {\bf R2.2} \grey{\emph{  I found the paper quite reader friendly but the authors can further improve this aspect, for instance with a section before the conclusion in which the authors lay down very simply and clearly the workflow for someone wanting to use their approach.}}\\

\noindent \textcolor{MyGreen}{\textbf{Addressed:}} The reviewer... \\

\noindent {\bf R2.3} \grey{\emph{ Relatedly, it's great that the authors make their dataset available, but they should also share their code (ideally, if possible, in the form a package).}}\\

\noindent \textcolor{MyGreen}{\textbf{Addressed:}} The reviewer... \\

\noindent {\bf R2.4} \grey{\emph{ The conclusion is currently quite short. There is scope for outlining more in detail the added value of the approach and how the data generated by it can be used in policy analysis. There is some discussion of these points in the conclusion, but it can be expanded.}}\\

\noindent \textcolor{MyGreen}{\textbf{Addressed:}} The reviewer... \\

\noindent {\bf R2.5} \grey{\emph{ I encourage the authors to further improve the clarity, especially at the end of the paper, to make their approach as accessible as possible to a non-technical audience.}}\\

\noindent \textcolor{MyGreen}{\textbf{Addressed:}} The reviewer... \\


\noindent \underline{\textbf{Response to Comments by Reviewer 3}}\\

\noindent {\bf R3.1} \grey{\emph{ While the analytics are very tempting I am greatly disappointed in the paper as it doesn?t really provide any substantive use of the method.}}\\

\noindent \textcolor{MyGreen}{\textbf{Addressed:}} The reviewer... \\

\noindent {\bf R3.2} \grey{\emph{ As it stands this manuscript doesn?t seek to make a theoretical contribution to the field even though the authors have at their disposal a unique dataset which I am sure would allow for several interesting theoretical questions to be asked.}}\\

\noindent \textcolor{MyGreen}{\textbf{Addressed:}} The reviewer... \\

\noindent {\bf R3.3} \grey{\emph{ Take out section 3. I don?t think it?s interesting to political scientists. Leave in an appendix or publish in a CS journal.}}\\

\noindent \textcolor{MyGreen}{\textbf{Addressed:}} The reviewer... \\

\noindent {\bf R3.4} \grey{\emph{ Use the space left by removing section 3 to propose a theoretical mechanism for policy diffusion and text reuse. This can include all the findings that you currently have ?and perhaps some others- ideology, policy area, geographic area, etc.}}\\

\noindent \textcolor{MyGreen}{\textbf{Addressed:}} The reviewer... \\

\noindent {\bf R3.5} \grey{\emph{ Reconsider the elements of section 4 moving elements of validity to an appendix or to the CS piece.}}\\

\noindent \textcolor{MyGreen}{\textbf{Addressed:}} The reviewer... \\

\noindent {\bf R3.6} \grey{\emph{ Properly present the results of your analysis based on the new theoretical argument.}}\\

\noindent \textcolor{MyGreen}{\textbf{Addressed:}} The reviewer... 


\end{document}
