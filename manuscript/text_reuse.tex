
%%%%%%%%%%%%%%%%%%%%%%%%%%%%%%%%%%%%%%%%%%%%%%%%%%%%%%%%%%%%%%%%%%%%%%%%%%%%%%%%%%%%%%%%%%
\documentclass[12pt]{article} %a4paper

\usepackage{graphicx} % to include pictures
\usepackage{subfig} % create subfigures within figures
\usepackage{pdflscape} % e.g. to rotate one page of the document
\usepackage{booktabs} % make better looking tabels with different line types and stuff
\usepackage[left=2.5cm,right=3cm,top=3cm,bottom=2.5cm]{geometry}
\usepackage{fancyhdr} % for pages with custom headers and footers
\usepackage[utf8]{inputenc}
\usepackage{float}
\usepackage{datetime}
\usepackage{natbib}
\usepackage{setspace}
\usepackage{amsmath}
\usepackage{hyperref}
\usepackage{pa}

\setlength{\parindent}{0.0in}
\setlength{\parskip}{1ex plus 0.5ex minus 0.2ex}
\mmddyyyydate
%%%%%%%%%%%%%%%%%%%%%%%%%%%%%%%%%%%%%%%%%%%%%%%%%%%%%%%%%%%%%%%%%%%%%%%%%%%%%%%%%%%%%%%%%%


\begin{document} 

\title{Identifying Equivalent State Bills through Text Reuse. \large Subtitle.}
\date{\today}
\author{}

\maketitle

\begin{abstract}
Much research has been focused on the diffusion of policy ideas in US state
legislatures. Most of this research uses hand coded data sets that identify
equivalent bills and analyze patterns of adoption of these bills. Bills on
equivalent policies often contain the same language, since legislators use past
legislation from other states or model legislation from interest groups as
templates when drafting new bills or model legislation from interest groups. In
this paper we evaluate the effectives of text reuse measures to detect bills
that address the same policy issues... 
\end{abstract}

\section{Introduction}






\bibliographystyle{chicago}
\bibliography{bibliography}


\end{document}











