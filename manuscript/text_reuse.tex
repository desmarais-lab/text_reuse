
%%%%%%%%%%%%%%%%%%%%%%%%%%%%%%%%%%%%%%%%%%%%%%%%%%%%%%%%%%%%%%%%%%%%%%%%%%%%%%%%%%%%%%%%%%
\documentclass[12pt]{article} %a4paper

\usepackage{graphicx} % to include pictures
\usepackage{subfig} % create subfigures within figures
\usepackage{pdflscape} % e.g. to rotate one page of the document
\usepackage{booktabs} % make better looking tabels with different line types and stuff
\usepackage[left=2.5cm,right=3cm,top=3cm,bottom=2.5cm]{geometry}
\usepackage{fancyhdr} % for pages with custom headers and footers
\usepackage[utf8]{inputenc}
\usepackage{float}
\usepackage{datetime}
\usepackage{natbib}
\usepackage{setspace}
\usepackage{amsmath}
\usepackage{hyperref}

\setlength{\parindent}{0.0in}
\setlength{\parskip}{1ex plus 0.5ex minus 0.2ex}
\mmddyyyydate
%%%%%%%%%%%%%%%%%%%%%%%%%%%%%%%%%%%%%%%%%%%%%%%%%%%%%%%%%%%%%%%%%%%%%%%%%%%%%%%%%%%%%%%%%%


\begin{document} 

\title{Identifying Equivalent State Bills through Text Reuse. \large Subtitle.}
\date{\today}
\author{}

\maketitle

\begin{abstract}
Much research has been focused on the diffusion of policy ideas in US state
legislatures. Most of this research uses hand coded data sets that identify
equivalent bills and analyze patterns of adoption of these bills. Bills on
equivalent policies often contain the same language, since legislators use past
legislation from other states or model legislation from interest groups as
templates when drafting new bills or model legislation from interest groups. In
this paper we evaluate the effectives of text reuse measures to detect bills
that address the same policy issues. We find that... 
\end{abstract}

\section{Introduction}

\subsection{Tasks}
\begin{itemize}
\item Write paragraph about how it would be great if text-reuse could be used to classify policy overlap (BD)
\item Write paragraph on why this is not easy (FL)
\end{itemize}
How ever, it is not clear, how well text reuse is suited to to detect real
policy diffusion. There are several complications that we are addressing in this
paper. First, every bill has procedural content that is not related to the
policy content of the bill. Since every state legislature has a set of such
standard or boiler plate text in each bill, there will be significant text reuse
between bills in the same state and possibly also between bills from different
states. Second, it is not obvious how much text reuse will mean substantive
policy overlap. Given that each bill pair has a continuous proportion of
overlapping text, setting the threshold too low might mean to classify bills as
equivalent that are only in the same policy area, or are on a similar issue, but
are opposed in content. On the other hand, setting the threshold too high could
mean overlooking equivalent bills because of small insignificant changes to the
text. 

In this paper we address these issues and evaluate how much policy overlap can
be detected using text reuse measures. Following Wilkerson et al. (2015) we use
supervised machine learning to separate boiler plate from substantive text. We
further more use an original data set on policy diffusion to evaluate how much
equivalent policy can be detected using measures of text reuse. Using the system
developed by Burges et al. (2015) we calculate text reuse scores for all pairs
of bills in our dataset. We then use these scores in a model that classifies
bills as equivalent and evaluate its performance with the validation data set. 

This work has several important implications. First, it allows us to estimate,
how policies are transferred between states. Do state legislators work mainly
from templates from other states or interest groups, or to what extent do
legislators draft their own bill text. Furthermore, text reuse is a relatively
simple metric to calculate for large amounts of text. Previous scholars of
policy diffusion mainly relied on case studies or manually coded data sets of
policy diffusion in few policy areas. If working copying text forms a
significant portion of how legislators adopt policies from other states, text
reuse can be used to easily gather comprehensive data sets on policy diffusion. 

[This might be overlapping a bit with your part]

\section{Background}

\subsection{Tasks}
\begin{itemize}
\item Policy diffusion (BD get papers)
\item Text re-use in CBP in bills 
\item Text re-use in general
\begin{itemize}
\item Plaigirism
\item Alignment
\end{itemize} 
\item Discussion of text-based classification
\end{itemize}


In order to detect text reuse between bills we follow \citet{wilderson2015tracing} and use the Smith-Waterman local alignment algorithm \citep{smith1981identification}. 

\section{Data}

We will rely on two main data sources to assess the reliability of text reuse to
identify substantively equivalent bills. 

In order to calculate the alignment scores between bills, we rely on a database collected by Burgess et al. (2015) and the Sunlight foundation. This data base contains approximately 500,000
bills from 2008 to 2015. The available metadata for the bills includes a timestamp for introduction and approval of the bill, the name and party affiliation of the sponsor(s), the state and the bill id. 

For the ideological matching, we rely on latent ideology scores measured by \citep{shor2011}. The data set contains scores for 20738 legislators from 50 state legislatures as well as their party affiliation and their time in office. 

We additionally construct a validation dataset of equivalent bills from information obtained from the National Conference of State Legislatures (NCSL). The NCSL publishes summary tables on specific policies, citing the relevant bills or sections in the state statutes of the states that have implemented regulations on this policy. We collected all these tables, and extracted all bills that address the same policy measure and that overlap with the time frame covered by our bill database.


\subsection{Tasks}
\begin{itemize}
\item Scrape Google urls for NCSL tables (FL) -- lower priority
    \begin{enumerate}
        \item Scraped 64 urls. 
        \item Can't get around the Google API limitation yet. Also get blocked when trying to scrape the search result.
    \end{enumerate}

\item Extract state  \& bill \# from tables (FL) -- lower priority
    \begin{enumerate}
        \item Didn't do this yet, let's first check how many are potentially suitable and then probably better to do by hand
    \end{enumerate}
\item See tasks in analysis -- create metadata file and dyadic file, then put them on the ACI (FL).
    \begin{enumerate}
        \item We have a preliminary dyadic file for all similar bill pairs and their alignment scores (this is from the LID approach: each bill is only compared to the 100 most similar ones)
        \item Matt transfering the database to a new server at the moment, it is therefore unavailable. Should be online again sometime today.
    \end{enumerate}
\item Store MALP data on ACI and make sure the identifiers match those in the bill data (FL). 
    \begin{enumerate}
        \item data is on aci
        \item Legislators can be matched by Name, State, Pary and Term. This information should be contained in the database. If not, the open state api has a legislator search method, that returns all necessary information.
    \end{enumerate}
\item Put policy diffusion data in project folder on ACI (BD).
\end{itemize}

\section{Analysis}
 Below we list X separate analyses designed to test the degree to which measuring text reuse measures policy overlap/diffusion. 

\begin{itemize}
\item We use statistical topic modeling to assess the major content areas represented by the text identified in the alignment algorithm. We consider whether the resultant topics align with policy areas.
\item The Measuring American Legislators Project (MALP) provides data that we will use to assess the significance of text re-use in US state legislation \citet{shor2011}. The most recent release of the MALP data covers 1993-2014. The MALP data provide ideological scores of legislators on an annual basis. We conduct a bill-level statistical network analysis to see whether the rate of bill-to-bill alignment is positively related to the ideological similarity of their sponsors.
\item We conduct a state-level analysis to see whether the volume of state-to-state text alignment is positively related to the policy diffusion networks inferred in \citet{desmarais2015}.
\end{itemize}


\subsection{Tasks}
\begin{itemize}
\item Develop bill metadata dataset (FL)
\begin{itemize}
\item unique bill identifier
\item sponsor identifier from  MALP data
\item state identifier
\item chamber identifier
\item Introduction date
\item Anything else, even if some is missing (status, committees, etc)
\end{itemize}
\item Develop bill-to-bill edgelists (FL)
\begin{itemize}
\item Sparse dyadic dataset (i.e., no observations when there is 0 alignment/overlap)
\item alignment score(s)
\item name of a text file in which aligned text is stored
\end{itemize}
\item Topic models, possibly of a relational form, applied to aligned text
\item Computationally intensive analysis of bill-to-bill alignment and sponsor ideology.
\item Analysis of alignment aggregated at the state level.

\end{itemize}


\bibliographystyle{chicago}
\bibliography{bibliography}


\end{document}











