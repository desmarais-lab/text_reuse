
%%%%%%%%%%%%%%%%%%%%%%%%%%%%%%%%%%%%%%%%%%%%%%%%%%%%%%%%%%%%%%%%%%%%%%%%%%%%%%%%%%%%%%%%%%
\documentclass[12pt]{article} %a4paper

\usepackage{graphicx} % to include pictures
\usepackage{subfig} % create subfigures within figures
\usepackage{pdflscape} % e.g. to rotate one page of the document
\usepackage{booktabs} % make better looking tabels with different line types and stuff
\usepackage[left=2.5cm,right=3cm,top=3cm,bottom=2.5cm]{geometry}
\usepackage{fancyhdr} % for pages with custom headers and footers
\usepackage[utf8]{inputenc}
\usepackage{float}
\usepackage{datetime}
\usepackage{natbib}
\usepackage{setspace}
\usepackage{amsmath}
\usepackage{hyperref}
\usepackage{pa}

\setlength{\parindent}{0.0in}
\setlength{\parskip}{1ex plus 0.5ex minus 0.2ex}
\mmddyyyydate
%%%%%%%%%%%%%%%%%%%%%%%%%%%%%%%%%%%%%%%%%%%%%%%%%%%%%%%%%%%%%%%%%%%%%%%%%%%%%%%%%%%%%%%%%%


\begin{document} 

\title{Identifying Equivalent State Bills through Text Reuse. \large Subtitle.}
\date{\today}
\author{}

\maketitle

\begin{abstract}
Much research has been focused on the diffusion of policy ideas in US state
legislatures. Most of this research uses hand coded data sets that identify
equivalent bills and analyze patterns of adoption of these bills. Bills on
equivalent policies often contain the same language, since legislators use past
legislation from other states or model legislation from interest groups as
templates when drafting new bills or model legislation from interest groups. In
this paper we evaluate the effectives of text reuse measures to detect bills
that address the same policy issues. We find that... 
\end{abstract}

\section{Introduction}

\subsection{Tasks}
\begin{itemize}
\item Write paragraph about how it would be great if text-reuse could be used to classify policy overlap (BD)
\item Write paragraph on why this is not easy (FL)
\end{itemize}
How ever, it is not clear, how well text reuse is suited to to detect real
policy diffusion. There are several complications that we are addressing in this
paper. First, every bill has procedural content that is not related to the
policy content of the bill. Since every state legislature has a set of such
standard or boiler plate text in each bill, there will be significant text reuse
between bills in the same state and possibly also between bills from different
states. Second, it is not obvious how much text reuse will mean substantive
policy overlap. Given that each bill pair has a continuous proportion of
overlapping text, setting the threshold too low might mean to classify bills as
equivalent that are only in the same policy area, or are on a similar issue, but
are opposed in content. On the other hand, setting the threshold too high could
mean overlooking equivalent bills because of small insignificant changes to the
text. 

In this paper we address these issues and evaluate how much policy overlap can
be detected using text reuse measures. Following Wilkerson et al. (2015) we use
supervised machine learning to separate boiler plate from substantive text. We
further more use an original data set on policy diffusion to evaluate how much
equivalent policy can be detected using measures of text reuse. Using the system
developed by Burges et al. (2015) we calculate text reuse scores for all pairs
of bills in our dataset. We then use these scores in a model that classifies
bills as equivalent and evaluate its performance with the validation data set. 

This work has several important implications. First, it allows us to estimate,
how policies are transferred between states. Do state legislators work mainly
from templates from other states or interest groups, or to what extent do
legislators draft their own bill text. Furthermore, text reuse is a relatively
simple metric to calculate for large amounts of text. Previous scholars of
policy diffusion mainly relied on case studies or manually coded data sets of
policy diffusion in few policy areas. If working copying text forms a
significant portion of how legislators adopt policies from other states, text
reuse can be used to easily gather comprehensive data sets on policy diffusion. 

[This might be overlapping a bit with your part]

\section{Background}

\subsection{Tasks}
\begin{itemize}
\item Policy diffusion (BD get papers)
\item Text re-use in CBP in bills 
\item Text re-use in general
\begin{itemize}
\item Plaigirism
\item Alignment
\end{itemize} 
\item Discussion of text-based classification
\end{itemize}

\section{Data}

We will rely on two main data sources to assess the reliability of text reuse to
identify substantively equivalent bills. 

One is a data base of bill text that was extracted through the sunlight
foundation's Open States API. This data base contains approximately 500,000
bills from 2008 to 2015. From this database we will extract the bills for which we
have ground truth on policy equivalence. 
Once we obtained these bills, we will use the local alignment algorithm
implemented by (Burgess et al. 2015) in order to obtain alignment scores for
these bills. 
Information on which bills implement equivalent policies is obtained from
several sources. One option is to rely on datasets compiled by scholars who have
studied policy diffusion in the past. Another option is to rely on materials
compiled by the National Conference of State Legislatures (NCSL). The NCSL keeps
track of specific policies and regulations across states. However, the tables
compiled by the NCSL most often refer to the relevant sections in the states'
statutes instead of to the bills that these sections are based on. In order to
use these material we would have to find a way to link statute sections to their
original bills. 


\subsection{Tasks}
\begin{itemize}
\item Scrape Google urls for NCSL tables (
\item Extract state  \& bill \# from tables (FL)
\end{itemize}

\section{Analysis}


\subsection{Tasks}
\begin{itemize}
\item Alternative validation options (BD)
\end{itemize}


\bibliographystyle{chicago}
\bibliography{bibliography}


\end{document}











