
%%%%%%%%%%%%%%%%%%%%%%%%%%%%%%%%%%%%%%%%%%%%%%%%%%%%%%%%%%%%%%%%%%%%%%%%%%%%%%%%%%%%%%%%%%
\documentclass[12pt]{article} %a4paper

\usepackage{graphicx} % to include pictures
\usepackage{subfig} % create subfigures within figures
\usepackage{pdflscape} % e.g. to rotate one page of the document
\usepackage{booktabs} % make better looking tabels with different line types and stuff
\usepackage[left=2.5cm,right=3cm,top=3cm,bottom=2.5cm]{geometry}
\usepackage{fancyhdr} % for pages with custom headers and footers
\usepackage[utf8]{inputenc}
\usepackage{float}
\usepackage{datetime}
\usepackage{natbib}
\usepackage{setspace}
\usepackage{amsmath}
\usepackage{hyperref}
\usepackage{pa}

\setlength{\parindent}{0.0in}
\setlength{\parskip}{1ex plus 0.5ex minus 0.2ex}
\mmddyyyydate
%%%%%%%%%%%%%%%%%%%%%%%%%%%%%%%%%%%%%%%%%%%%%%%%%%%%%%%%%%%%%%%%%%%%%%%%%%%%%%%%%%%%%%%%%%


\begin{document} 

\title{Identifying Equivalent State Bills through Text Reuse. \large Subtitle.}
\date{\today}
\author{}

\maketitle

\begin{abstract}
Much research has been focused on the diffusion of policy ideas in US state
legislatures. Most of this research uses hand coded data sets that identify
equivalent bills and analyze patterns of adoption of these bills. Bills on
equivalent policies often contain the same language, since legislators use past
legislation from other states or model legislation from interest groups as
templates when drafting new bills or model legislation from interest groups. In
this paper we evaluate the effectives of text reuse measures to detect bills
that address the same policy issues. We find that... 
\end{abstract}

\section{Introduction}

\subsection{Tasks}
\begin{itemize}
\item Write paragraph about how it would be great if text-reuse could be used to classify policy overlap (BD)
\item Write paragraph on why this is not easy (FL)
\item Summarize what we do to solve problem and validate (FL)
\end{itemize}

...

\section{Background}

\subsection{Tasks}
\begin{itemize}
\item Policy diffusion (BD get papers)
\item Text re-use in CBP in bills 
\item Text re-use in general
\begin{itemize}
\item Plaigirism
\item Alignment
\end{itemize} 
\item Discussion of text-based classification
\end{itemize}

\section{Data}

We will rely on two main data sources to assess the reliability of text reuse to
identify substantively equivalent bills. 

One is a data base of bill text that was extracted through the sunlight
foundation's Open States API. This data base contains approximately 500,000
bills from 2008 to 2015. From this database we will extract the bills for which we
have ground truth on policy equivalence. 
Once we obtained these bills, we will use the local alignment algorithm
implemented by (Burgess et al. 2015) in order to obtain alignment scores for
these bills. 
Information on which bills implement equivalent policies is obtained from
several sources. One option is to rely on datasets compiled by scholars who have
studied policy diffusion in the past. Another option is to rely on materials
compiled by the National Conference of State Legislatures (NCSL). The NCSL keeps
track of specific policies and regulations across states. However, the tables
compiled by the NCSL most often refer to the relevant sections in the states'
statutes instead of to the bills that these sections are based on. In order to
use these material we would have to find a way to link statute sections to their
original bills. 


\subsection{Tasks}
\begin{itemize}
\item Scrape Google urls for NCSL tables (FL)
\item Extract state  \& bill \# from tables (FL)
\end{itemize}

\section{Analysis}


\subsection{Tasks}
\begin{itemize}
\item Alternative validation options (BD)
\end{itemize}


\bibliographystyle{chicago}
\bibliography{bibliography}


\end{document}











